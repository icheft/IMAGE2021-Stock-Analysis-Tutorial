\documentclass{beamer}
\usepackage[utf8]{inputenc}

\usepackage{utopia} %font utopia imported
\usepackage{xcolor}
\usepackage{xparse}
\usepackage{xeCJK}
\usepackage{graphicx}
\usepackage{fontspec} 
\newfontfamily\DejaSans{DejaVu Sans}

\setCJKmainfont{PingFang TC}

\usetheme{Frankfurt}
\usecolortheme{seagull} % beaver

\definecolor{pinky}{RGB}{219, 48, 122}

\newcommand{\code}[1]{\texttt{\textcolor{pinky}{#1}}}

\usebackgroundtemplate{
\includegraphics[width=\paperwidth,
height=\paperheight]{Page_169}
}

%------------------------------------------------------------
%This block of code defines the information to appear in the
%Title page
\title{進階 Python - 試教}

\subtitle{用 Python 簡單打造一個屬於自己的投資分析 App}

\author{Brian L. Chen}

\institute{
  Department of Information Management\\
  National Taiwan University
}

\date{IMAGE Camp, \today}

% \logo{\includegraphics[height=1.5cm]{lion-logo.jpg}}

%End of title page configuration block
%------------------------------------------------------------



%------------------------------------------------------------
%The next block of commands puts the table of contents at the 
%beginning of each section and highlights the current section:

\AtBeginSection[]
{
  \begin{frame}
    \frametitle{目錄}
    \tableofcontents[currentsection]
  \end{frame}
}
%------------------------------------------------------------


\begin{document}

%The next statement creates the title page.
\frame{\titlepage}


%---------------------------------------------------------
%This block of code is for the table of contents after
%the title page
\begin{frame}
  \frametitle{目錄}
  \tableofcontents
\end{frame}
%---------------------------------------------------------

\section{課程基本資訊}

%---------------------------------------------------------
\begin{frame}
  \begin{itemize}
    \item 課程名稱:進階 Python - 製作自己的第一個股市分析 App
    \item 內容概述:實際撰寫可運作的 Python 爬蟲程式並且運用他人寫好的套件寫出一個簡單但實用的 App
    \item 時長:3 小時
    \item 教材:\code{Python}、範例都在 \href{https://github.com/icheft/IMAGE2021-Stock-Analysis-Tutorial}{GitHub},請自行取用~
  \end{itemize}

\end{frame}

%---------------------------------------------------------

\section{接下來三小時內你會學到什麼?}
\begin{frame}
  \begin{itemize}
    \item Python Package 介紹
    \item 網路爬蟲 / 抓取 Yahoo! Finance 資料
    \item 介紹 Python 畫圖工具
          \begin{itemize}
            \item \code{matplotlib}
            \item 其他第三方工具
          \end{itemize}
    \item 介紹 \href{https://streamlit.io}{Streamlit}
    \item 實作
  \end{itemize}
\end{frame}

\section{進入正題}
\begin{frame}
  \frametitle{網路爬蟲?}
  網路爬蟲是可以自動化替你蒐集網頁上資訊的程式

\end{frame}

\begin{frame}
  \frametitle{Motivation}

  試著想想看,如果你需要各公司的財報資訊來做股市的研究,該怎麼做呢?\\
  打開瀏覽器,輸入「公開資訊觀測站」的網址,下條件篩選出想找的財報,在下載成 CSV 檔,然後再進行分析?\footnotemark\\
  \quad \\
  以上流程就要花你十幾二十分鐘!\\
  如果要同時分析好幾間公司呢?\\
  2018 年?2017 年?
  \footnotetext[1]{資料來源:\href{https://medium.com/@bindaguo/網路爬蟲淺談-afcae0694f13}{Medium 文章 - 網路爬蟲淺談}}
\end{frame}

\begin{frame}
  \frametitle{Python 爬蟲工具}
  這堂課會需要你安裝幾個 Python 套件,大家可以把它們想像成是善心人士已經寫好的程式,
  它們讓你可以直接使用套件,完成爬蟲任務。

  \begin{itemize}
    \item \code{bs4}: 最大宗爬蟲套件
    \item \code{pandas}: 分析資料最好用
  \end{itemize}

\end{frame}

\begin{frame}
  \frametitle{這麼重要的功能當然有人幫你寫好了 - \code{yfinance}}
  \begin{itemize}
    \item \code{bs4} 可以幫我們實現爬蟲,但每次我們要查股價都要從頭爬蟲還是太慢了
    \item 一定有更好的解決方式
    \item 使用 API?使用套件?
  \end{itemize}
\end{frame}

\begin{frame}
  \frametitle{這麼重要的功能當然有人幫你寫好了 - \code{yfinance}}

  Yahoo deprecated their Finance API in 2017. So you can see many websites talking about alternatives for Yahoo Finance API. However, the python library \code{yfinance} offers a temporary fix to the problem by scraping the data from Yahoo! Finance and returning the data in the DataFrame format. So you can still use Yahoo Finance to get free stock market data\footnotemark.

  \footnotetext[2]{資料來源:\href{https://towardsdatascience.com/free-stock-data-for-python-using-yahoo-finance-api-9dafd96cad2e}{Medium 文章 - Yahoo Finance API}}

\end{frame}

\end{document}
