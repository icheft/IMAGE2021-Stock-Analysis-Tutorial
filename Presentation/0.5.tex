\documentclass{beamer}
\usepackage[utf8]{inputenc}

\usepackage{utopia} %font utopia imported
\usepackage{fontspec} 
\usepackage{xcolor}
\usepackage{xparse}
\usepackage{xeCJK}
\usepackage{graphicx}
\setCJKmainfont{PingFang TC}

\usetheme{Frankfurt}
\usecolortheme{seagull} % beaver

\definecolor{pinky}{RGB}{219, 48, 122}

\newcommand{\code}[1]{\texttt{\textcolor{pinky}{#1}}}

\usebackgroundtemplate{
\includegraphics[width=\paperwidth,
height=\paperheight]{Page_169}
}

%------------------------------------------------------------
%This block of code defines the information to appear in the
%Title page
\title{進階 Python - 0.5 驗}

\subtitle{用 Python 簡單打造一個屬於自己的投資分析 App}

\author{Brian L. Chen}

\institute{
  Department of Information Management\\
  National Taiwan University
}

\date{IMAGE Camp, \today}

% \logo{\includegraphics[height=1.5cm]{lion-logo.jpg}}

%End of title page configuration block
%------------------------------------------------------------



%------------------------------------------------------------
%The next block of commands puts the table of contents at the 
%beginning of each section and highlights the current section:

\AtBeginSection[]
{
  \begin{frame}
    \frametitle{目錄}
    \tableofcontents[currentsection]
  \end{frame}
}
%------------------------------------------------------------


\begin{document}

%The next statement creates the title page.
\frame{\titlepage}


%---------------------------------------------------------
%This block of code is for the table of contents after
%the title page
\begin{frame}
\frametitle{目錄}
\tableofcontents
\end{frame}
%---------------------------------------------------------

\section{課程基本資訊}

%---------------------------------------------------------
\begin{frame}
  \begin{itemize}
    \item 名稱:進階 Python - 製作自己的第一個股市分析 App\footnotemark
    \item 內容概述:用一個簡單但實用的 App 例子教導高中生一些 Python 能夠做到的事
    \item 時長:希望可以達到 \alert{\textbf{3 小時}}
    \item 教材:\code{Python}、屆時範例都會在 GitHub 上面\footnotemark
  \end{itemize}
\footnotetext[1]{待訂}
\footnotetext[2]{學員不用會 GitHub,但助教可能需要知道}

\end{frame}

%---------------------------------------------------------

\section{授課動機}

%---------------------------------------------------------
%Highlighting text
\begin{frame}
  本教學主要目標為讓同學知道 Python 可以在短時間內如何做出一個類似 App 的工具。\\
  以目前最為流行的股票當作媒介,讓高中生也有機會將資訊與股票市場結合,
  藉由簡單的分析,給自己未來投資或是家人更多 insights 。
  
  \vspace{5mm} %5mm vertical space
  
  有鑒於教學時程短暫,本課僅能作為初步介紹用,如果要學習更多相關資源到時候也會公布一份完整的學習清單。
\end{frame}

\begin{frame}
  \frametitle{有別於上一屆的地方}
  上一屆主要是教 \code{function}。此部分很重要沒錯,但希望可以放在\textbf{基礎 Python}就好。

  \vspace{5mm} %5mm vertical space
  
  如果真的要教學的話,這邊也可以提供幾個方向:
  \begin{examples}
    用 \href{https://livebook.manning.com/book/tiny-python-projects/chapter-11/}{11 Bottles of Beer Song: Writing and testing functions} 來舉例 Function 的重要。應該可以很快就講完。
  \end{examples}

  進階 Python 想讓同學感受一下自己做一個 App 的感覺,讓程式可以活生生的展現在他人眼前。
\end{frame}
%---------------------------------------------------------

\section{課程內容安排}
\begin{frame}
\begin{itemize}
  \item Python Package 介紹
  \item 網路爬蟲 / 抓取 Yahoo! Finance 資料
  \item 介紹 Python 畫圖工具           
  \begin{itemize}
    \item \code{matplotlib}
    \item 其他第三方工具
  \end{itemize}
  \item 介紹 \href{https://streamlit.io}{Streamlit}                     
  \item 實作                               
\end{itemize}
\end{frame}

\section{上課形式}
\begin{frame}
\begin{itemize}
  \item 主要以實作為主。過程中可能會有許多地方會是初學者比較不懂的,不過沒關係。因為課程僅有 1.5 小時,
  很多地方無法一次融會貫通是正常的。學員只要看得懂哪裡是需要修改的即可。
  \item 基本上會有類似 template 的東西提供給學員參考。
  \item 所以就是一些簡單的投影片 + Live coding 🔥
\end{itemize}
\end{frame}

\section{對TA的要求}
\begin{frame}
  \begin{itemize}
  \item 對 TA 基本沒什麼要求。只要會一點 Python,知道 \code{pandas.DataFrame} 怎麼用即可。
  \item TA 要求人數目前仍不確定
  \item 如果會使用 \textit{GitHub} 更好
  \end{itemize}
\end{frame}
\section{希望課程部能夠協助的部分}
\begin{frame}
  目前要能夠確立以下幾點  
  \begin{itemize}
    \item 電腦教室的 \textbf{VS Code} 是可以正常運行的
    \item 教室的電腦環境可以使用如 \code{pip install blah-blah-blah} 的指令\footnotemark
  \end{itemize}
  \footnotetext[3]{因為講者並非使用 Windows 作業系統,在上面不確定怎麼完整運行 Unix 終端機指令}
\end{frame}



\end{document}
