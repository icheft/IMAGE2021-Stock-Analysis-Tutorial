\documentclass{beamer}
\usepackage[utf8]{inputenc}

\usepackage{utopia} %font utopia imported
\usepackage{xcolor}
\usepackage{xparse}
\usepackage{xeCJK}
\usepackage{graphicx}
\usepackage{fontspec} 
\newfontfamily\DejaSans{DejaVu Sans}

\setCJKmainfont{PingFang TC}

\usetheme{Frankfurt}
\usecolortheme{seagull} % beaver

\definecolor{pinky}{RGB}{219, 48, 122}

\newcommand{\code}[1]{\texttt{\textcolor{pinky}{#1}}}

\usebackgroundtemplate{
\includegraphics[width=\paperwidth,
height=\paperheight]{Page_169}
}

%------------------------------------------------------------
%This block of code defines the information to appear in the
%Title page
\title{進階 Python - 2 驗}

\subtitle{用 Python 簡單打造一個屬於自己的投資分析 App}

\author{Brian L. Chen}

\institute{
  Department of Information Management\\
  National Taiwan University
}

\date{IMAGE Camp, \today}

% \logo{\includegraphics[height=1.5cm]{lion-logo.jpg}}

%End of title page configuration block
%------------------------------------------------------------



%------------------------------------------------------------
%The next block of commands puts the table of contents at the 
%beginning of each section and highlights the current section:

\AtBeginSection[]
{
  \begin{frame}
    \frametitle{目錄}
    \tableofcontents[currentsection]
  \end{frame}
}
%------------------------------------------------------------


\begin{document}

%The next statement creates the title page.
\frame{\titlepage}


%---------------------------------------------------------
%This block of code is for the table of contents after
%the title page
\begin{frame}
\frametitle{目錄}
\tableofcontents
\end{frame}
%---------------------------------------------------------

\section{課程基本資訊}

%---------------------------------------------------------
\begin{frame}
  \begin{itemize}
    \item 名稱:進階 Python - 製作自己的第一個股市分析 App
    \item 內容概述:用一個簡單但實用的 App 例子教導高中生一些 Python 能夠做到的事
    \item 時長:希望可以達到 \alert{\textbf{3 小時}}
    \item 教材:\code{Python}、屆時範例都會在 GitHub 上面
  \end{itemize}

\end{frame}

%---------------------------------------------------------

\section{授課動機}

%---------------------------------------------------------
%Highlighting text
\begin{frame}
  本教學主要目標為讓同學知道 Python 可以在短時間內如何做出一個類似 App 的工具。\\
  以目前最為流行的股票當作媒介,讓高中生也有機會將資訊與股票市場結合,
  藉由簡單的分析,給自己未來投資或是家人更多 insights 。
  
  \vspace{5mm} %5mm vertical space
  
  有鑒於教學時程短暫,本課僅能作為初步介紹用,如果要學習更多相關資源到時候也會公布一份完整的學習清單。
\end{frame}

\begin{frame}
  \frametitle{你會學到什麼?}
  進階 Python 想讓同學感受一下自己做一個 App 的感覺,讓程式可以活生生的展現在他人眼前。

  所以本課程結束後,同學有機會
  
\end{frame}
%---------------------------------------------------------

\section{課程內容安排}
\begin{frame}
\begin{itemize}
  \item Python Package 介紹
  \item 網路爬蟲 / 抓取 Yahoo! Finance 資料
  \item 介紹 Python 畫圖工具           
  \begin{itemize}
    \item \code{matplotlib}
    \item 其他第三方工具
  \end{itemize}
  \item 介紹 \href{https://streamlit.io}{Streamlit}                     
  \item 實作                               
\end{itemize}
\end{frame}

\section{進入正題}
\begin{frame}
    \frametitle{網路爬蟲?}
網路爬蟲是可以自動化替你蒐集網頁上資訊的程式

\end{frame}

\begin{frame}
    \frametitle{Motivation}

    試著想想看,如果你需要各公司的財報資訊來做股市的研究,該怎麼做呢?\\
    打開瀏覽器,輸入「公開資訊觀測站」的網址,下條件篩選出想找的財報,在下載成 CSV 檔,然後再進行分析?\footnotemark\\
    \quad \\ 
    以上流程就要花你十幾二十分鐘!\\
    如果要同時分析好幾間公司呢?\\
    2018 年?2017 年?
    \footnotetext[3]{資料來源:https://medium.com/@bindaguo/網路爬蟲淺談-afcae0694f13}
\end{frame}

\begin{frame}
    \frametitle{Python 爬蟲工具}
    這堂課會需要你安裝幾個 Python 套件,大家可以把它們想像成是善心人士已經寫好的程式,
    它們讓你可以直接使用套件,完成爬蟲任務。

    \begin{itemize}
        \item \code{bs4}: 最大宗爬蟲套件
        \item \code{yfinance}: 已經有人寫好 Yahoo! Finance 的爬蟲軟體啦!
        \item \code{pandas}: 分析資料最好用
    \end{itemize}

\end{frame}

\begin{frame}
    
    {\DejaSans 😁} 讓我們轉移到 Jupyter Notebook 吧!

\end{frame}




\end{document}
